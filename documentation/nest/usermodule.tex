\section{User Module}

\subsection*{Overview}

The \textbf{UserModule} manages user-specific metadata and preferences that are not handled by Firebase Auth. It stores and retrieves information such as first name, last name, birthdate, and contact data using Firestore.

\subsection*{Responsibilities}

\begin{itemize}
    \item Create and update user profile information in Firestore.
    \item Retrieve authenticated user metadata (from Firebase Auth).
    \item Provide a minimal contact endpoint for support messages (placeholder).
\end{itemize}

\subsection*{Dependencies}

\begin{itemize}
    \item \texttt{Firestore} - Used to store extended user info.
    \item \texttt{AuthService} - Used to fetch Firebase Auth metadata (e.g., display name, email).
\end{itemize}

\subsection*{Key Service Methods}

\begin{itemize}
    \item \texttt{createUserInfo(uid, dto)}: Creates a Firestore document with full name and email pulled from Firebase Auth.
    \item \texttt{updateUserInfo(uid, dto)}: Updates the user info with merge semantics and a timestamp.
    \item \texttt{getUserInfo(uid)}: Returns user info from Firestore, or \texttt{null} if not found.
\end{itemize}

\subsection*{Exposed Routes}

\subsubsection*{POST /user/me}

\begin{tabular}{>{\bfseries}l l}
\toprule
Method & POST \\
Route & /user/me \\
Auth required & Yes \\
Status codes & 201 Created, 400 Bad Request, 401 Unauthorized \\
\bottomrule
\end{tabular}

\textbf{Description:} Creates the Firestore user profile based on the authenticated user's UID and metadata from Firebase Auth.

\vspace{1em}
\textbf{Payload Example:}
\begin{verbatim}
{
  "birthdate": "2000-01-01"
}
\end{verbatim}

\subsubsection*{PATCH /user/me}

\begin{tabular}{>{\bfseries}l l}
\toprule
Method & PATCH \\
Route & /user/me \\
Auth required & Yes \\
Status codes & 200 OK, 400 Bad Request, 401 Unauthorized \\
\bottomrule
\end{tabular}

\textbf{Description:} Updates user information in Firestore. Uses \texttt{merge: true} semantics to preserve existing fields.

\vspace{1em}
\textbf{Payload Example:}
\begin{verbatim}
{
  "firstname": "Jane",
  "lastname": "Smith",
  "email": "jane@example.com"
}
\end{verbatim}

\subsubsection*{POST /user/contact}

\begin{tabular}{>{\bfseries}l l}
\toprule
Method & POST \\
Route & /user/contact \\
Auth required & No \\
Status codes & 200 OK (placeholder) \\
\bottomrule
\end{tabular}

\textbf{Description:} Endpoint for users to contact admin/support. Currently a stub (to be implemented).

\vspace{1em}
\textbf{Payload Example:}
\begin{verbatim}
{
  "email": "user@example.com",
  "message": "I need help with..."
}
\end{verbatim}

\subsection*{Testing}

Unit tests validate:

\begin{itemize}
    \item Creation of user info from Firebase metadata.
    \item Updating user info with merge semantics.
    \item Retrieval of user data (existing and non-existing cases).
\end{itemize}

Mocks are used for:

\begin{itemize}
    \item \texttt{AuthService.getUserByUid()}
    \item Firestore collection, doc, set, and get operations
\end{itemize}
